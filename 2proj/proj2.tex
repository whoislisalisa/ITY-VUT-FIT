\documentclass[a4paper, 11pt, twocolumn]{article}
\usepackage[utf8]{inputenc}
\usepackage[czech]{babel} 
\usepackage[IL2]{fontenc}
\usepackage[left=1.4cm,text={18.2cm, 25.2cm},top=2.3cm]{geometry} 
\usepackage{csquotes}
\usepackage[unicode]{hyperref}
\usepackage{times}
\usepackage{amsthm}
\usepackage{amsmath}
\usepackage{amsthm}
\usepackage{amssymb}
\usepackage{setspace}
\usepackage{array}
%%\vsize=1.6180\hsize

\begin{document}
    \begin{titlepage}
        \begin{center}
            \textsc{\Huge Vysoké učení technické v~Brně \break \vspace{0.3em} Fakulta informačních technologií \\}
            \vspace{\stretch{0.382}}
            {\LARGE Typografie a publikování\,--\,2. projekt \\ \vspace{0.3em} Sazba dokumentů a matematických výrazů}
    \end{center}
     \vspace{\stretch{0.618}}
        {\Large 2023 \hfill Frederika Kmeťová (xkmeto00)}
    \end{titlepage}

    \newpage
    \section*{Úvod}
    V této úloze si vyzkoušíme sazbu titulní strany, mate\-matických vzorců, prostředí a dalších textových struktur obvyklých pro technicky zaměřené texty – například Defi- nice 1 nebo rovnice (3) na straně 1. Pro vytvoření těchto odkazů používáme kombinace příkazů \verb|\label|, \verb|\ref|, \verb|\eqref| a \verb|\pageref|. Před odkazy patří nezlomitelná mezera. Pro zvýrazňování textu jsou zde několikrát použity příkazy \verb|\verb| a \verb|\emph|.
    \par Na titulní straně je použito prostředí \verb|titlepage| a sá\-zení nadpisu podle optického středu s využitím \textit{přesného} zlatého řezu. Tento postup byl probírán na přednášce. Dále jsou na titulní straně použity čtyři různé velikosti písma a mezi dvojicemi řádků textu je použito odřádkování se zadanou relativní velikostí 0,5\,em a 0,4\,em\footnote[1]{Nezapomeňte použít správný typ mezery mezi číslem a jednotkou.}.

    \section{Matematický text}
    V této sekci se podíváme na sázení matematických sym\-bolů a výrazů v plynulém textu pomocí prostředí \verb|math|. Definice a věty sázíme pomocí příkazu \verb|\newtheorem| s využitím balíku \verb|amsthm|. Někdy je vhodné použít kon\-strukci \verb|${}$| nebo \verb|\mbox{}|, která říká, že (matematický) text nemá být zalomen.
    
    \newtheorem*{definition}{Definice 1}
        \begin{definition}
            \label{def:1}
            \emph{Zásobníkový automat} (ZA) je definován jako sedmice tvaru $A = (Q, \Sigma, \Gamma, \delta, q_0, Z_0, F$), kde:
        \end{definition}
            \begin{itemize}
                \item $Q$ \textit{je konečná množina} vnitřních (řídicích) stavů,
                \item $\Sigma$ \textit{je konečná} vstupní abeceda,
                \item $\Gamma$ \textit{je konečná} zásobníková abeceda,
                \item $\delta$ \emph{je} přechodová funkce $Q \times (\Sigma \cup \{\epsilon\}) \times  \Gamma \rightarrow  2^{Q\times\Gamma ^\ast}$,
                \item $q_0 \in  Q$ \emph{je} počáteční stav, $Z_0 \in \Gamma$  je startovací symbol zásobníku \emph{a} $F \subseteq Q$ \emph{je množina} koncových stavů.
            \end{itemize}
            \par Nechť $P = (Q, \Sigma, \Gamma, \delta, q_0, Z_0, F$) je ZA. \emph{Konfigurací} nazveme trojici $(q, w, \alpha) \in Q \times \Sigma^\ast \times\Gamma^\ast$, kde $q$ je aktuální stav vnitřního řízení, $w$ je dosud nezpracovaná část vstupního řetězce a $\alpha = Z_{i_1}Z_{i_2} \ldots Z_{i_k} $ je obsah zásobníku.
            \subsection{Podsekce obsahující definici a větu}
            \begin{definition}
                \label{def:2}
                \emph{Řetězec} $w$ \emph{nad abecedou} $\Sigma$ \emph{je přijat ZA} $A$~jestliže $(q_0, w, Z_0) \underset{A}{\overset{\ast}{\vdash}}$~$(q_F, \epsilon, \gamma)$ pro nějaké $\gamma \in \Gamma^\ast$ a $q_F \in F$. 
                Množinu\,$L(A) =$ \{$w\;|\;w$ je přijat ZA $A$\} $\subseteq \Sigma^\ast $ je \emph{jazyk přijímaný ZA} $A$.
            \end{definition}
    \newtheorem{veta}{Věta}
    \begin{veta}
    Třída jazyků, které jsou přijímány ZA, odpovídá \emph{bezkontextovým jazykům.}
    \end{veta}

    \section{Rovnice} 
    Složitější matematické formulace sázíme mimo plynulý text pomocí prostředí \verb|displaymath|. Lze umístit i ně\-kolik výrazů na jeden řádek, ale pak je třeba tyto vhodně oddělit, například příkazem \verb|\quad|. 
    S
    %% rovnica DOPLNIT
    \newline V rovnici (2) jsou využity tři typy závorek s různou \emph{ex\-plicitně} definovanou velikostí. Také nepřehlédněte, že na- sledující tři rovnice mají zarovnaná rovnítka, a použijte k tomuto účelu vhodné prostředí.
    %% rovnica DOPLNIT
    \newline V této větě vidíme, jak vypadá implicitní vysázení limity $\lim_{n \to \infty}f(n)$ v~normálním odstavci textu. Podobně je to i s~dalšími symboly jako
\end{document}